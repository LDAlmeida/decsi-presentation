\begin{frame}{Problema de pesquisa}
Em uma entrevista ao programa "Bom Dia, Ministra",a ministra Margareth Menezes citou que a cada R\$ 100 investidos pelo Governo Federal, apenas R\$ 0,57 é investido em cultura.

\vspace{\baselineskip}

Dados do Observatório Itaú Cultural apontam que, em 2020, a economia da cultura e das indústrias criativas (Ecic) do Brasil movimentou R\$ 230,14 bilhões, equivalente a 3,11\% do PIB — superando o da indústria automobilística.

\vspace{\baselineskip}

Dessa forma, como viabilizar a criação de uma plataforma digital que promova a cultura local e valorize os talentos artísticos de João Monlevade?

\end{frame}

\begin{sidepic}{beamerthemesrc/assets/casa}{A casa de cultura}
A Casa de Cultura de João Monlevade, inaugurada em 27 de novembro
de 1981, desempenha um papel essencial no fomento à cultura local. 

\vspace{\baselineskip}

No entanto, ela não possui uma plataforma adequada para divulgar o trabalho de artistas locais de forma integrada e acessível.
\end{sidepic}

\begin{frame}{Objetivo geral}

Desenvolver uma plataforma digital que funcione
como uma vitrine para os talentos artísticos de Monlevade, contribuindo para a promoção e
valorização da cultura local.

\end{frame}

\begin{frame}{Objetivos específicos}
Os objetivos específicos são:

\begin{itemize}
  \item Desenvolvimento de um site com sistema CMS;
  \item Desenvolvimento de um sistema de gestão para a Escola de Artes;
  \item Criação da Vitrine de Artistas;
  \item Implementação do módulo de Editais;
  \item Avaliação das funcionalidades implementadas.
\end{itemize}
\end{frame}

\begin{frame}{Metodologia}
Os passos para execução deste trabalho foram assim definidos:

\begin{itemize}
  \item Revisar Unified Modeling Language (UML);
  \item Desenvolvimento do sistema CMS para publicação de artigos; \item Desenvolvimento do sistema de gestão para a Casa de Cultura;
  \item Testar o sistema.
\end{itemize}
\end{frame}

\begin{frame}{Revisao bibliografica}
\textbf{Tecnologias utilizadas:}

\vspace{\baselineskip}

\begin{center}
  \includegraphics[height=0.2\textheight]{beamerthemesrc/assets/python.jpg}
  \includegraphics[height=0.2\textheight]{beamerthemesrc/assets/django-logo-positive.png}
  \includegraphics[height=0.2\textheight]{beamerthemesrc/assets/wagtail4298.jpg}
\end{center}
  
  \end{frame}

\begin{frame}{Revisao bibliografica}

O Wagtail CRX é um sistema de gerenciamento de conteúdo baseado no framework Django e no CMS Wagtail. Ele suporta práticas de SEO avançadas, possibilita a personalização e extensão das suas funcionalidades, suporta múltiplos idiomas e principalmente, possui uma interface intuitiva e amigável.

\vspace{\baselineskip}

O Django admin oferece facilidade na modificação de informações, além de acesso direto aos dados
guardados no banco de dados do site, o que será importante para as funcionalidades da Escola de Artes.


\vspace{\baselineskip}

\end{frame}
